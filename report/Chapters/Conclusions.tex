% Chapter Template

\chapter{Conclusion} % Main chapter title

\label{SectionX} % Change X to a consecutive number; for referencing this chapter elsewhere, use \ref{SectionX}

\lhead{Section 4. \emph{Conclusion}} % Change X to a consecutive number; this is for the header on each page - perhaps a shortened title

%----------------------------------------------------------------------------------------
%	SECTION 1
%----------------------------------------------------------------------------------------

In the applications of wavelet transform such as image denoising and image inpainting, careful choice of a suitable filter bank as well as coefficients such as threshold values and decomposition values can be very essential for obtaining a desirable result. Theoretically, frames are an over-complete version of a basis set and the tight frames are regarded as an over-complete version of an orthogonal basis set. 

In practical, the algorithms and computing methods used for different filter banks are basically the same, while the implementation with the tight framelet method can result in a certain amount of redundancy. However, this redundancy property of a tight framelet is desirable in the application of image inpainting since it gives a robustness in the transform and the errors consequently become less destructive. In the context of image denoising, the redundancy property seems not so effective as it behaves in the inpainting. Moreover, tight framelet filter banks are proved to be undesirable in denoising especially when the noise level is low (As shown in table 3.1).

\vfill
